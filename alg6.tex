
\documentclass{article}%{ctexart}
\usepackage{CJKutf8}
%\usepackage[UTF8]{CTeX}
%\usepackage{xeCJK} %调用 xeCJK 宏包
%\setCJKmainfont{SimSun} %设置 CJK 主字体为 SimSun (宋体

\usepackage{amsmath,amssymb,amsthm,color,mathrsfs}
\usepackage{enumerate,anysize}%
\usepackage{qrcode,fancyhdr}
\usepackage{commath}
%\pd,\dif,\od,...
\pagestyle{fancy}
\lhead{201821130049}%\rhead{段奎元}
\marginsize{1in}{1in}{1in}{1in}%

%%Settheory notation 
\theoremstyle{remark}
\newtheorem*{claim*}{Claim}
\newtheorem{lemma}{Lemma}
\DeclareMathOperator{\ot}{ordertype}
\DeclareMathOperator{\dom}{dom}
\DeclareMathOperator{\ran}{ran}
\DeclareMathOperator{\Ord}{Ord}
%\DeclareMathOperator{\sup}{sup}
%\DeclareMathOperator{\inf}{inf}

%\def\a#1{\mathbb{#1}}%代数常用的双体
\def\p#1{\mathscr{#1}} %概率论常用的手写体
\def\ae{\textrm{ a.e.}} \def\aeto{\xrightarrow{\ae}}
\def\pp{\mathbb{P}}
%微分几何符号用commath包的命令
%\def\t{T\!\!}%微分几何用切空间
%\def\d{\,\mathrm{d}}
\def\r{\mathbb{R}}
\newcommand{\cc}{\mathfrak{c}}
\def\pow#1{\mathscr P #1}

\def\<{\langle}
\def\>{\rangle}
\def\res{\!\!\upharpoonright\!\!_}%refine the restriction
\def\bcup{\textstyle\bigcup}
\def\bcap{\textstyle\bigcap}
\def\bsum{\textstyle\sum}
\def\bprod{\textstyle\prod}
\def\Oset{\varnothing}
\def\diff{\mathbin{{\Delta}}}

\def\email#1{{\texttt{#1}}}
\def\prob{\par\color{blue}\item}
\def\soln{\color{black}\par\noindent\underline{\sc Solution}:\hspace*{1em}\parindent=2em}
\newcommand{\isep}[1][0pt]{\addtolength{\itemsep}{#1}}

\AtBeginDocument{\begin{CJK}{UTF8}{gbsn}}
\AtEndDocument{\end{CJK}}
%%%重新使用cjkutf8,因仅在姓名使用中文,章节定理等仍用英文%
%\begin{document}%在document后定义含中文的命令
\def\asemail{段奎元\\ %% 姓名
	SID: 201821130049\\ %% 学号
	\email{dkuei@outlook.com}} %% 电邮}aUTHORsIDeMAIL
%尝试不用这些命令
%# -*- coding:utf-8 -*-
%! Mode:: "TeX:UTF-8"
%!TEX encoding = UTF-8 Unicode 
%!TEX TS-program = pdflatex
%%%%%%%%
\begin{document}
\title{Algebra HW6}
\author{\asemail}
\maketitle

%按照题号抄题并解答。
\begin{enumerate}
	\isep[1em]

\prob
\textit{Page 133,3 }%8,10,23,24共5题
If $R$ is a ring and $a\in R$, then $J=\{r\in R \mid ra=0\}$ is a left ideal and 
$K=\{r\in R \mid ar=0\}$ is a right ideal in $R$.
\soln
For any $r \in R$, $i,j\in J$, $k,l\in K$, 
\begin{align*}
	(ri)a & =r(ia) =r0=0, \\
	(i-j)a & =ia-ja=0-0=0; \\
	a(kr) & =(ak)r =0r=0, \\
	a(k-l) & =ak-al=0-0=0. 
\end{align*}
Thus $ri \in J, i-j \in J$, $J$ is a left ideal; $kr \in K, k-l \in K$, $K$ is a right ideal.
\qed

\prob
\textit{Page 133,8 }
Let $R$ be a ring with identity and $S$ the ring of all $n\times n$ matrices over $R$. 
$J$ is an ideal of $S$ if and only if $J$ is the ring of all $n\times n$ matrices over $I$ for some ideal $I$ in $R$. 

[Hint: Given $J$, let $I$ be the set of all those elements of $R$ that appear as the row $1$-column $1$ entry of some matrix in $J$. 
Use the matrices $E_{r,s}$, where $1 \leq r,s \leq n$, and $E_{r,s}$ has $1_R$ as the row $r$-column $s$ entry and $0$ elsewhere. 
Observe that for a matrix $A = (a_{ij}), E_{p,r}AE_{s,q}$ is the matrix with $a_{rs}$ the row $p$-column $q$ entry and $0$ elsewhere.]
\soln
\begin{itemize}
	\item[``$\Rightarrow$'':] Consider the set $I_{rs}=\{a\in R \mid aE_{rs}\in J\}$.
	Since the elementary matrix switching the $i$-th and $j$-th row or column $P_{ij}\in R$, $I_{rs}=\{a\in R \mid aE_{rs}\in J\}=\{a\in R \mid P_{1s}aE_{rs}P_{1r}\in J\}=I_{11}$. Thus denote $I=I_{rs}$ for all $1 \leq r,s \leq n$. 
	$I$ is an ideal since for any $a,b\in I, x\in R$, 
	\begin{align*}
		(a-b)E_{11} & = aE_{11}-bE_{11} \in J, \\
		(xa)E_{11} & = xE_{11} aE_{11} \in J, \\
		(ax)E_{11} & = aE_{11} xE_{11} \in J. 
	\end{align*}
	In addition, $J$ is the ring of all $n\times n$ matrices over $I$ since for any $1 \leq r,s \leq n$, $(a_{ij}) \in J$, $a_{rs}E_{rs}=(a_{ij})E_{rs} \in J$ infers $a_{rs}\in I_{rs}=I$. 
	\item[``$\Leftarrow$'':] 
	For any $(a_{ij}),(b_{ij}) \in J$ where $a_{ij},b_{ij} \in I$ for $1 \leq i,j \leq n$. 
	For any $(x_{ij})\in R$, 
	\begin{align*}
		(a_{ij})-(b_{ij}) & = ((a-b)_{ij}) \in J, \\
		(a_{ij})(x_{ij}) & = (\sum_{k=1}^n a_{ik}x_{kj}) \in J, \\
		(x_{ij})(a_{ij}) & = (\sum_{k=1}^n x_{ik}a_{kj}) \in J.
	\end{align*}
	Thus $J$ is an ideal of $S$.
\end{itemize}
\qed

\prob
\textit{Page 133,10}
\begin{enumerate}
	\item Show that $Z$ is a principal ideal ring [see \textbf{Theorem I.3.1}].
	\item Every homomorphic image of a principal ideal ring is also a principal ideal ring.
	\item $Z_m$ is a principal ideal ring for every $m > 0$.
\end{enumerate}
\soln
\begin{enumerate}
	\item By \textbf{Theorem I.3.1}, every subgroup of $Z$ has the form $\<m\>$ for some $m\in Z$. Since every ideal of $Z$ must be subgroup of $Z$, it must have the form $\<m\>$ and is a principal ideal.
	In addition, $\forall n\in Z$, $km \in \<m\>$, $nkm=(nk)m \in \<m\>$. Since $Z$ is communitative, $\<m\>$ is an ideal of $Z$. Thus every ideal of $Z$ is of the form $\<m\>$.
	\item Assume $P$ is a principal ideal ring, $R$ is a ring, $\phi$ is a homomorphism from $P$ to $R$; then $\phi(P)$ is a ring. For any ideal $I$ in $\phi(P)$, $\forall a,b \in \phi^{-1}(I), x \in P$, \begin{align*}
		\phi(a-b) & = \phi(a)-\phi(b) \in I, \\
		\phi(xa) & = \phi(x)\phi(a) \in I, \\
		\phi(ax) & = \phi(a)\phi(x) \in I. 
	\end{align*}
	Thus $\phi^{-1}(I)$ is an ideal in $P$. For any ideal $J$ in $P$, $\forall a,b \in J, x \in P$,
	\begin{align*}
		\phi(a)-\phi(b) & =\phi(a-b) \in \phi(J), \\
		\phi(x)\phi(a) & =\phi(xa) \in \phi(J), \\
		\phi(a)\phi(x) & =\phi(ax) \in \phi(J). 
	\end{align*}
	Thus $\phi(J)$ is an ideal in $\phi(P)$.

	Suppose $\phi^{-1}(I)=\<m\>$ for some $m\in P$, then $\phi^{-1}(\<\phi(m)\>)$ is an ideal in $P$ and $m\in \phi^{-1}(\<\phi(m)\>)$. Since $\phi^{-1}(I)\<m\>=\phi^{-1}(\<\phi(m)\>)$, we get $I=\<\phi(m)\>$ is a principal ideal and $\phi(P)$ is a principal ideal ring.
	\item $Z_m\cong Z/mZ$, construct a map \begin{align*}
		\phi:\quad Z &\to Z_m \\
		z &\mapsto [z]_m
	\end{align*}
	For any $a,b\in Z$, \begin{align*}
		\phi(a)+\phi(b) &= [a]_m+[b]_m=[a+b]_m, \\
		\phi(a)\phi(b) &= [a]_m[b]_m=[ab]_m,
	\end{align*}
	thus $\phi$ is an epimorphism. By conclusions in the above two, $Z_m=\phi(Z)$ is a principal ideal ring.
\end{enumerate}
\qed

\newpage
\prob\textit{Page 133,23}
An element $e$ in a ring $R$ is said to be idempotent if $e^2 = e$. 
An element of the center of the ring $R$ is said to be central. 
If $e$ is a central idempotent in a ring $R$ with identity, then
\begin{enumerate}
	\item $1_R - e$ is a central idempotent;
	\item $eR$ and $(1_R - e)R$ are ideals in $R$ such that $R = eR \times (1_R - e)R$.
\end{enumerate}
\soln
\begin{enumerate}
	\item For any $a\in R$, 
	\begin{align*}
		a(1_R-e) & = a-ae =a-ea =(1_R-e)a; \\
		(1_R-e)^2 & = 1_R-e-e+e^2 =1_R-e.
	\end{align*}
	Thus $1_R - e$ is a central idempotent.
	\item For any central idempotent $f\in R$, $fa,fb \in fR$ and $x \in R$, \begin{align*}
		fa-fb & = f(a-b) \in fR, \\
		x(fa) & = (xf)a =(fx)a=f(xa) \in fR, \\
		(fa)x & = f(ax) \in fR. 
	\end{align*}
	Thus $fR$ is an ideal and so do $eR,(1_R - e)R$. 
	
	For any $r\in R$, $r=er+(1_R -e)r\in eR+(1_R -e)R$ thus $R=eR+(1_R -e)R$. 
	For $a\in eR \cap (1_R - e)R$, suppose $a=ex=(1_R -e)y$ for some $x,y\in R$. \begin{align*}
		a &= ea+(1_R -e)a \\
		&= e(1_R -e)y+(1_R -e)ex \\
		&= (e-e^2)y+(e-e^2)x \\
		&= 0.
	\end{align*}
	Therefore, $R = eR \times (1_R - e)R$.
\end{enumerate}\qed

\prob\textit{Page 133,24}
Idempotent elements $e_1, \cdots ,e_n$ in a ring $R$ [see Exercise 23] are said to be orthogonal if $e_ie_j = 0$ for $i \neq j$. If $R, R_1, \cdots, R_n$ are rings with identity, then the following conditions are equivalent:
\begin{enumerate}
	\item $R \cong R_1 \times \cdots \times R_n$.
	\item $R$ contains a set of orthogonal central idempotents [Exercise 23] $\{e_1, \cdots , e_n\}$ such that $e_l + e_2 + \cdots + e_n = 1_R$ and $e_iR \cong R_i$ for each i.
	\item $R$ is the internal direct product $R = A_1 \times \cdots \times A_n$ where each $A_i$ is an ideal of $R$ such that $A_i \cong R_i$.

	[Hint: (a)$\to$(b): The elements $f_1 = (1_{R_1},0, \cdots ,0), f_2 = (0,1_{R_2}, \cdots ,0)$, ,$ f_n = (0, \cdots ,0,1_{R_n})$ 
	arf orthogonal cfntral idfmpotfnts in $S = R_1 \times \cdots \times R_n$ such that $f_1 + \cdots + f_n = 1_s$ and $f_iS\cong R_i$. 
	(b) $\to$ (c) Note that $A_k = e_k R$ is the principal ideal $\<e_k\>$ in $R$ and that $e_kR$ is itself a ring with identity $e_k$.]
\end{enumerate}
\soln
\begin{itemize}
	\item[(a)$\Rightarrow$(b):] Let $S=R_1 \times \cdots \times R_n$, the elements $e_1 = (1_{R_1},0, \cdots ,0), e_2 = (0,1_{R_2}, \cdots ,0), \cdots , e_n = (0, \cdots ,0,1_{R_n})$ are orthogonal since $e_i e_j \in R_i \cap R_j =\{0\}$ for $i\neq j$. $1_S=\sum_{k=1}^n e_k$. For any $1\leq i\leq n$, $a=\sum_{k=1}^n e_k a_k\in S$, 
	\begin{align*}
		e_i a &= e_i a_i = a_i e_i = a e_i,\\
		e_i^2 &= (0,\cdots,1_{R_i}^2,\cdots,0)= e_i.
	\end{align*}
	Thus $e_i$ is a central idempotent. Since $R\cong S$, $R$ also contains a set of orthogonal central idempotents satisfies the conditions.
	\item[(b)$\Rightarrow$(c):] Let $A_i=e_i R\cong R_i$, then $R= A_1 \times \cdots \times A_n$.
	\item[(c)$\Rightarrow$(a):] It suffices to show $R_1\times R_2\cong A_1 \times A_2$ where the products are both external products by \textbf{Theorem 2.24}. Suppose the isomorphism $\phi_1: R_1 \to A_1$ and $\phi_2: R_2 \to A_2$, and define
	\begin{align*}
		\phi:\quad R_1\times R_2 &\to A_1 \times A_2 \\
		(r_1,r_2) &\mapsto (\phi_1(r_1),\phi_2(r_2)).
	\end{align*}
	$\phi$ is an isomorphism between groups. For any $(r_1,r_2),(s_1,s_2)\in R_1\times R_2$, \begin{align*}
		(\phi_1(r_1),\phi_2(r_2))(\phi_1(s_1),\phi_2(s_2))&=(\phi_1(r_1)\phi_1(s_1),\phi_2(r_2)\phi_2(s_2)) \\
		&= (\phi_1(r_1s_1),\phi_2(r_2s_2)) \\
		&= \phi((r_1s_1,r_2s_2)),
	\end{align*}
	so $\phi$ is an isomorphism between rings. $R= A_1 \times \cdots \times A_n\cong  R_1 \times \cdots \times R_n$.
\end{itemize}

\qed

%$|\frac{x^m-1}{x-1}-m|<\varepsilon$
\end{enumerate}
\end{document}