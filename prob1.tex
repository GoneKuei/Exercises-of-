
\documentclass{article}%{ctexart}
\usepackage{CJKutf8}
%\usepackage[UTF8]{CTeX}
%\usepackage{xeCJK} %调用 xeCJK 宏包
%\setCJKmainfont{SimSun} %设置 CJK 主字体为 SimSun (宋体

\usepackage{amsmath,amssymb,amsthm,color,mathrsfs}
\usepackage{enumerate,anysize}%
\usepackage{qrcode,fancyhdr}
\usepackage{commath}
%\pd,\dif,\od,...
\pagestyle{fancy}
\lhead{201821130049}%\rhead{段奎元}
\marginsize{1in}{1in}{1in}{1in}%

%%Settheory notation 
\theoremstyle{remark}
\newtheorem*{claim*}{Claim}
\newtheorem{lemma}{Lemma}
\DeclareMathOperator{\ot}{ordertype}
\DeclareMathOperator{\dom}{dom}
\DeclareMathOperator{\ran}{ran}
\DeclareMathOperator{\Ord}{Ord}
%\DeclareMathOperator{\sup}{sup}
%\DeclareMathOperator{\inf}{inf}

%\def\a#1{\mathbb{#1}}%代数常用的双体
\def\p#1{\mathscr{#1}} %概率论常用的手写体
\def\ae{\textrm{ a.e.}} \def\aeto{\xrightarrow{\ae}}
\def\pp{\mathbb{P}}
%微分几何符号用commath包的命令
%\def\t{T\!\!}%微分几何用切空间
%\def\d{\,\mathrm{d}}
\def\r{\mathbb{R}}
\newcommand{\cc}{\mathfrak{c}}
\def\pow#1{\mathscr P #1}

\def\<{\langle}
\def\>{\rangle}
\def\res{\!\!\upharpoonright\!\!_}%refine the restriction
\def\bcup{\textstyle\bigcup}
\def\bcap{\textstyle\bigcap}
\def\bsum{\textstyle\sum}
\def\bprod{\textstyle\prod}
\def\Oset{\varnothing}
\def\diff{\mathbin{{\Delta}}}

\def\email#1{{\texttt{#1}}}
\def\prob{\par\color{blue}\item}
\def\soln{\color{black}\par\noindent\underline{\sc Solution}:\hspace*{1em}\parindent=2em}
\newcommand{\isep}[1][0pt]{\addtolength{\itemsep}{#1}}

\AtBeginDocument{\begin{CJK}{UTF8}{gbsn}}
\AtEndDocument{\end{CJK}}
%%%重新使用cjkutf8,因仅在姓名使用中文,章节定理等仍用英文%
%\begin{document}%在document后定义含中文的命令
\def\asemail{段奎元\\ %% 姓名
	SID: 201821130049\\ %% 学号
	\email{dkuei@outlook.com}} %% 电邮}aUTHORsIDeMAIL
%尝试不用这些命令
%# -*- coding:utf-8 -*-
%! Mode:: "TeX:UTF-8"
%!TEX encoding = UTF-8 Unicode 
%!TEX TS-program = pdflatex
%%%%%%%%
\begin{document}
\title{Probability Theory HW3}
\author{\asemail}
\maketitle

%按照题号抄题并解答。
\begin{enumerate}
	\isep[1em]
	\prob
	Construct a counter example to show that if $\mu$ is not $\sigma$ finite, its extension from a semialgebra to the minimal $\sigma$-algebra may not be unique.
	\soln
	The semialgebra $\p S=\{(a,b]:a\leq b\}$, $\mu((a,b])=0$ if $(a,b]\cap (0,1]=\Oset$ and $\mu((a,b])=\infty$ if $(a,b]\cap (0,1]\neq\Oset$. Then the extension $\mu(\{1\})$ can be $0$ or $\infty$.

	\prob
	$\p S$ is a semialgebra, $\mu$ is a finite measure on $\p S$, the triple $(\Omega, \p A^*, \mu^*)$ is the completion of extension of $\mu$ on $\sigma(\p S)$. Let
		\begin{align*}
			\mu_*(A) & =\sup\{\bsum_n \mu(A_n): A_n\in\p S \text{ pairwise disjoint, }\bsum_n A_n\subset A\},
			\\ \p A_*&=\{A\subset \Omega:\mu^*(A)=\mu_*(A)\}.
		\end{align*}
		Prove that $\p A^*\supset\p A_*$.
	\soln
	$\forall A\in\p A_*,D\subset \Omega$,
		\begin{align*}
			\mu^*(A\cap D) & =\inf\{\bsum_{n=1}^\infty \mu(B_n): B_n\in\p S,A\cap D\subset \bcup_{n=1}^\infty B_n\}
			\\ \mu^*(A^c\cap D) & =\inf\{\bsum_{n=1}^\infty \mu(C_n): C_n\in\p S,A^c\cap D\subset \bcup_{n=1}^\infty C_n\}
			\\ &=\inf\{\bsum_{n=1}^\infty \mu(C_n): C_n\in\p S,\bcap_{n=1}^\infty C_n^c\subset A\cup D^c\}
			\\ &=
		\end{align*}
		\medskip

		\prob
		Let $\p C=\{C_{a,b}=[-b,a)\cup(a,b]:0<a<b\}$, and define the measure $\mu(C_{a,b})=b-a$. Prove that $\mu$ can extend to a measure on $\sigma(\p C)$. Is $[1,2]$ a $\mu^*$-measurable set?
		\soln
		Let $\p S=\p C\cup \{\r, \Oset\}$,
		\begin{enumerate}
			\item $\{\r, \Oset\}\subset \p S$
			\item $\forall A,B\in \p S$, if $A=C_{a,b},B=C_{c,d}\in \p C$, $C_{a,b}\cap C_{c,d}=C_{\max{a,c},\min{b,d}}\in \p C\subset \p S$. In the case either of $A,B\in\{\r, \Oset\}$, $A\cap B\in \p S$ is trivial.
			\item $\forall A,A_1\in \p S, A_1\subset A$, the case $A,A_1\in\{\r,\Oset\}$ is trivial. If $A=C_{a,b},A_1=C_{a_1,b_1}$, then $A\setminus A_1=C_{a,a_1}+C_{b_1,b}$.
		\end{enumerate}
		so $\p S$ is a semialgebra. Assign $\mu$ on $\r,\Oset$ to $\infty,0$ respectively, and then $\mu$ is a measure on $\p S$ analogue of the semialgebra on $\r$. By \textbf{Theorem 1.42} $\mu$ has a unique extension.

		For $C_{1,2}\subset \r$, $\mu^*(C_{1,2})=1$ and $\mu^*([1,2])=\mu^*([-2,-1])=1$, i.e.$\mu^*(C_{1,2})\neq \mu^*([1,2]\cap C_{1,2})+\mu^*([-2,-1]^c\cap C_{1,2})$. $[1,2]$ is not a $\mu^*$-measurable set.
		\prob
		Let $f:\r \ni x\mapsto \frac{x}{3}\in \r, A_0=[0,1]$. Then the sequence of $A_{n+1}=f(A_n)\cup (\frac{2}{3}+f(A_n))(n\geq 0)$ is monotone decreasing. The limitation of $A_n$ is called Cantor set, $C=\bcap_n A_n$. Prove the Lebesgue measure of Cantor set is 0.
		\soln
		It suffices to prove the outer measure of $C$ is 0. $\forall n\geq 0$, $f(A_n) \cap (\frac{2}{3}+f(A_n))=\Oset$ and $A_n$ is sum of some intervals since $f(A_n)\subset f(A_0)=[0,\frac{1}{3}]$. Then
	\begin{align*}
		\mu(A_{n+1}) & = \mu(f(A_n))+ \mu((\frac{2}{3}+f(A_n))) & \text{finite additivity}
		\\ & = \frac{2}{3}\mu(A_n) & \text{length of interval}
		\\ & = (\frac{2}{3})^{n+1} & \text{inductively}.
	\end{align*}
	This follows
	\begin{align*}
		\mu(C)=\lim_{n\to \infty}\mu(A_n)=\lim_{n\to \infty}(\frac{2}{3})^n=0,
	\end{align*}
	by the continuity of measure.


\end{enumerate}
\end{document}