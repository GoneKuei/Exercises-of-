
\documentclass{article}%{ctexart}
\usepackage{CJKutf8}
%\usepackage[UTF8]{CTeX}
%\usepackage{xeCJK} %调用 xeCJK 宏包
%\setCJKmainfont{SimSun} %设置 CJK 主字体为 SimSun (宋体

\usepackage{amsmath,amssymb,amsthm,color,mathrsfs}
\usepackage{enumerate,anysize}%
\usepackage{qrcode,fancyhdr}
\usepackage{commath}
%\pd,\dif,\od,...
\pagestyle{fancy}
\lhead{201821130049}%\rhead{段奎元}
\marginsize{1in}{1in}{1in}{1in}%

%%Settheory notation 
\theoremstyle{remark}
\newtheorem*{claim*}{Claim}
\newtheorem{lemma}{Lemma}
\DeclareMathOperator{\ot}{ordertype}
\DeclareMathOperator{\dom}{dom}
\DeclareMathOperator{\ran}{ran}
\DeclareMathOperator{\Ord}{Ord}
%\DeclareMathOperator{\sup}{sup}
%\DeclareMathOperator{\inf}{inf}

%\def\a#1{\mathbb{#1}}%代数常用的双体
\def\p#1{\mathscr{#1}} %概率论常用的手写体
\def\ae{\textrm{ a.e.}} \def\aeto{\xrightarrow{\ae}}
\def\pp{\mathbb{P}}
%微分几何符号用commath包的命令
%\def\t{T\!\!}%微分几何用切空间
%\def\d{\,\mathrm{d}}
\def\r{\mathbb{R}}
\newcommand{\cc}{\mathfrak{c}}
\def\pow#1{\mathscr P #1}

\def\<{\langle}
\def\>{\rangle}
\def\res{\!\!\upharpoonright\!\!_}%refine the restriction
\def\bcup{\textstyle\bigcup}
\def\bcap{\textstyle\bigcap}
\def\bsum{\textstyle\sum}
\def\bprod{\textstyle\prod}
\def\Oset{\varnothing}
\def\diff{\mathbin{{\Delta}}}

\def\email#1{{\texttt{#1}}}
\def\prob{\par\color{blue}\item}
\def\soln{\color{black}\par\noindent\underline{\sc Solution}:\hspace*{1em}\parindent=2em}
\newcommand{\isep}[1][0pt]{\addtolength{\itemsep}{#1}}

\AtBeginDocument{\begin{CJK}{UTF8}{gbsn}}
\AtEndDocument{\end{CJK}}
%%%重新使用cjkutf8,因仅在姓名使用中文,章节定理等仍用英文%
%\begin{document}%在document后定义含中文的命令
\def\asemail{段奎元\\ %% 姓名
	SID: 201821130049\\ %% 学号
	\email{dkuei@outlook.com}} %% 电邮}aUTHORsIDeMAIL
%尝试不用这些命令
%# -*- coding:utf-8 -*-
%! Mode:: "TeX:UTF-8"
%!TEX encoding = UTF-8 Unicode 
%!TEX TS-program = pdflatex
%%%%%%%%
\begin{document}

\title{Probability Theory HW5}
\author{\asemail}
\maketitle

%按照题号抄题并解答。
\begin{enumerate}
	\isep[1em]

\prob
\textit{Page 46,13 Property 2.36}%14(2), 16, 22, 24
\begin{enumerate}
	\item If $f_n \aeto f$, then every subsequence $\{f_{n_k}\}$ satisfies $f_{n_k} \aeto f$.
	\item If $f_n \aeto f, f_n \aeto f^\prime$, then $f=f^\prime \ae$.
	\item If $f_n \aeto f, g_n=f_n \ae, f=g \ae$, then $g_n \aeto g$.
	\item If $f_n^{(k)} \aeto f^{(k)},k=1,\cdots,m$, $g\in C(\bar{\r}^m)$, then $$
	g(f_n^{(1)},\cdots,f_n^{(m)})\aeto g(f^{(1)},\cdots,f^{(m)}).$$
\end{enumerate}
\soln
\begin{enumerate}
	\item Since $f_n \aeto f$ there exists a null set $N$ s.t. $\forall x\in \Omega\setminus N, f_n(x)\to f(x)$. 
	Thus for every subsequence $\{f_{n_k}\}, f_{n_k}(x)\to f(x)$, i.e. $f_{n_k} \aeto f$.
	\item There exist two null sets $N,N^\prime$ such that 
	$\forall x\in \Omega\setminus N, f_n(x)\to f(x)$, and 
	\\ $\forall x\in \Omega\setminus N^\prime, f_n(x)\to f^\prime(x)$. 
	Thus $\forall x\in \Omega\setminus (N\cup N^\prime), f_n(x)\to f(x), f_n(x)\to f^\prime(x)$; then $f(x)= f^\prime(x)$ for any $x\in \Omega\setminus (N\cup N^\prime)$. 
	Since $\mu(N\cup N^\prime)=0$, we get $f=f^\prime \ae$.
	\item There exist null sets $N_1,N_2,N_3$ for $f_n \aeto f, g_n=f_n \ae, f=g \ae$ respectively. 
	For any $x\in \Omega\setminus (\cup_{i=1}^3 N_i), f_n(x) \to f(x), g_n(x)=f_n(x) , f(x)=g(x)$, so $g_n(x) \to g(x)$. 
	Since $\mu(\cup_{i=1}^3 N_i)=0$, we get $g_n \aeto g$.
	\item Denote $N_k=\{x\in \Omega^{(k)} \mid f_n^{(k)}(x) \not\to f^{(k)}(x) \} $, and we have $\mu(N_k)=0$ by $f_n^{(k)} \aeto f^{(k)},k=1,\cdots,m$.
	For any $(x^{(1)},\cdots,x^{(m)})\in \prod_{k=1}^m \Omega^{(k)}\setminus N_k,$ 
	$$\left(f_n^{(1)} (x^{(1)} ), \cdots, f_n^{(m)} (x^{(m)} )\right)\to \left(f^{(1)} (x^{(1)} ), \cdots, f^{(m)} (x^{(m)} )\right).$$
	Since $\mu(\prod_{k=1}^m \Omega^{(k)}\setminus N_k)=\mu(\prod_{k=1}^m \Omega^{(k)})$, the convergence keeps $\ae$
	By continuity of $g$, we get $$
	g(f_n^{(1)},\cdots,f_n^{(m)})\aeto g(f^{(1)},\cdots,f^{(m)}).$$
\end{enumerate}

\prob
\textit{Page 47,14 Theorem 2.38(2) }%14(2), 16, 22, 24

Suppose $f,f_n,n\geq 1$ are finite measurable functions. Then $f_n-f_m\aeto 0$ if and only if $$
\forall \varepsilon >0,\mu(\cap^\infty_{n=1}\cup^\infty_{v=1}\{|f_{n+v}-f_n |\geq \varepsilon\})=0. $$
Specially when $\mu$ is finite, $f_n-f_m\aeto 0$ if and only if $$
\forall \varepsilon >0,\mu(\cup^\infty_{v=1}\{|f_{n+v}-f_n |\geq \varepsilon\})\to 0(n\to\infty). $$
\soln 
$f_n-f_m\aeto 0$ if and only if $|f_{n+v}-f_n|\aeto 0$ by letting $n=\min\{m,n\}, v=|m-n|$. Notice 
$\{|f_{n+v}-f_n |\geq \varepsilon\}=\{x: |f_{n+v}-f_n |\geq \varepsilon\}$.
\begin{itemize}
	\item[``$\Rightarrow$'':] 
	There exists a null set $A$ such that $\forall \varepsilon >0$, there exists $N\in \mathbb{N}_+$ s.t. $\forall n\geq N, v\in  \mathbb{N}_+, x\in \Omega\setminus A, |f_{n+v}(x)-f_n(x) |<\varepsilon$.
	Then $\mu(\cup^\infty_{n=N}\cup^\infty_{v=1}\{x: |f_{n+v}(x)-f_n(x) |\geq \varepsilon\})=0$, and apparently $$
	\mu(\cap^\infty_{n=1}\cup^\infty_{v=1}\{|f_{n+v}-f_n |\geq \varepsilon\})= 0,
	 \mu(\cup^\infty_{v=1}\{|f_{n+v}-f_n |\geq \varepsilon\})\to 0(n\to\infty).$$
	\item[``$\Leftarrow$'':] 
	When $\mu$ is finite, $\mu(\cup^\infty_{v=1}\{|f_{n+v}-f_n |\geq \varepsilon\})\to 0(n\to\infty)$ means $\forall \phi>0 \exists N$ s.t. $\mu(\cup^\infty_{v=1}\{|f_{n+v}-f_n |\geq \varepsilon\})< \phi$ for all $n\geq N$.
	Then $\mu(\cap^\infty_{n=1}\cup^\infty_{v=1}\{|f_{n+v}-f_n |\geq \varepsilon\})=0$ for all $\varepsilon >0$. 
\vspace{2em}
\end{itemize}


\prob\textit{Page 47,16}
Let $\xi_n=1_{A_n}$, then $\xi_n \xrightarrow{\pp} 0$ if and only if $\pp(A_n)\to 0$.
\soln For any $\varepsilon > 0$, $\{|\xi_n|\geq\varepsilon\}=\{A_n\}$.
\begin{align*}
	\xi_n \xrightarrow{\pp} 0 &\Leftrightarrow \forall \varepsilon >0,\pp(|\xi_n|\geq\varepsilon)\to 0(n\to \infty) \\
	&\Leftrightarrow \forall \varepsilon >0,\pp(A_n)\to 0(n\to \infty) 
\end{align*}

\prob\textit{Page 47,22}
For any random variable sequence $\xi_n$, there is a positive integer sequence $a_n$ s.t. $a_n \xi_n \xrightarrow{\pp} 0.$
\soln $\xi_n \xrightarrow{\pp} 0 \Leftrightarrow \forall \varepsilon,\phi >0\exists N>0, \pp(|\xi_n|\geq\varepsilon)< \phi, \forall n\geq N$
\vspace{3em}

\prob\textit{Page 47,24}
Prove two theorems 2.49 and 2.50.
\begin{enumerate}
	\item[Theorem 2.49] If $\xi_n-\xi^\prime_n\xrightarrow{\pp} 0$ and $\xi^\prime_n\xrightarrow{d} \xi$, then $\xi_n\xrightarrow{d} \xi$.
	\item[Theorem 2.50] If $\xi_n\xrightarrow{d} \xi,\eta_n\xrightarrow{d}a(\mathrm{const})$, then $\xi_n+\eta_n\xrightarrow{d}\xi+a$. 
\end{enumerate}
\soln \vspace{5em}

\end{enumerate}\end{document}