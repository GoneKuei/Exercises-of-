
\documentclass{article}%{ctexart}
\usepackage{CJKutf8}
%\usepackage[UTF8]{CTeX}
%\usepackage{xeCJK} %调用 xeCJK 宏包
%\setCJKmainfont{SimSun} %设置 CJK 主字体为 SimSun (宋体

\usepackage{amsmath,amssymb,amsthm,color,mathrsfs}
\usepackage{enumerate,anysize}%
\usepackage{qrcode,fancyhdr}
\usepackage{commath}
%\pd,\dif,\od,...
\pagestyle{fancy}
\lhead{201821130049}%\rhead{段奎元}
\marginsize{1in}{1in}{1in}{1in}%

%%Settheory notation 
\theoremstyle{remark}
\newtheorem*{claim*}{Claim}
\newtheorem{lemma}{Lemma}
\DeclareMathOperator{\ot}{ordertype}
\DeclareMathOperator{\dom}{dom}
\DeclareMathOperator{\ran}{ran}
\DeclareMathOperator{\Ord}{Ord}
%\DeclareMathOperator{\sup}{sup}
%\DeclareMathOperator{\inf}{inf}

%\def\a#1{\mathbb{#1}}%代数常用的双体
\def\p#1{\mathscr{#1}} %概率论常用的手写体
\def\ae{\textrm{ a.e.}} \def\aeto{\xrightarrow{\ae}}
\def\pp{\mathbb{P}}
%微分几何符号用commath包的命令
%\def\t{T\!\!}%微分几何用切空间
%\def\d{\,\mathrm{d}}
\def\r{\mathbb{R}}
\newcommand{\cc}{\mathfrak{c}}
\def\pow#1{\mathscr P #1}

\def\<{\langle}
\def\>{\rangle}
\def\res{\!\!\upharpoonright\!\!_}%refine the restriction
\def\bcup{\textstyle\bigcup}
\def\bcap{\textstyle\bigcap}
\def\bsum{\textstyle\sum}
\def\bprod{\textstyle\prod}
\def\Oset{\varnothing}
\def\diff{\mathbin{{\Delta}}}

\def\email#1{{\texttt{#1}}}
\def\prob{\par\color{blue}\item}
\def\soln{\color{black}\par\noindent\underline{\sc Solution}:\hspace*{1em}\parindent=2em}
\newcommand{\isep}[1][0pt]{\addtolength{\itemsep}{#1}}

\AtBeginDocument{\begin{CJK}{UTF8}{gbsn}}
\AtEndDocument{\end{CJK}}
%%%重新使用cjkutf8,因仅在姓名使用中文,章节定理等仍用英文%
%\begin{document}%在document后定义含中文的命令
\def\asemail{段奎元\\ %% 姓名
	SID: 201821130049\\ %% 学号
	\email{dkuei@outlook.com}} %% 电邮}aUTHORsIDeMAIL
%尝试不用这些命令
%# -*- coding:utf-8 -*-
%! Mode:: "TeX:UTF-8"
%!TEX encoding = UTF-8 Unicode 
%!TEX TS-program = pdflatex
%%%%%%%%
\begin{document}
\title{Algebra HW3}
\author{\asemail}
\maketitle

%按照题号抄题并解答。
\begin{enumerate}
\isep[1em]

\prob
	\textit{Page 111,1} 
	\begin{enumerate}
		\item Find a normal series of $D_4$ consisting of 4 subgroups.
		\item Find all composition series of the group $D_4$. \label{allc}
		\item Do part \eqref{allc}  for the group $A_4$.
		\item Do part \eqref{allc}  for the group $S_3\times Z_2$.
		\item Find all composition factors of $S_4$ and $D_6$. 
	\end{enumerate}
\soln
\begin{enumerate}
	\item $D_4>Z_4>Z_2>\{e\}$, it is also a composition series.
	\item The only composition factor of $D_4$ is $Z_2$ according to the last series. By applying the Jordan-H\"{o}lder Theorem, the composition series are $D_4>Z_4>Z_2>\{e\}$, $D_4>K_4>Z_2>\{e\}$.
	\item $A_4$
	\item $S_3\times Z_2=D_6$ has $Z_2,Z_3$ as its simple normal subgroups.

	$S_3\times Z_2>S_3>Z_3>\{e\}$,

	$S_3\times Z_2>Z_6>Z_3>\{e\}$,

	$S_3\times Z_2>S_3>Z_2>\{e\}$, 
	
	$S_3\times Z_2>Z_6>Z_2>\{e\}$.
	\item The composition factors of $D_6$, according to the last composition series of $S_3\times Z_2=D_6$, are $Z_2,Z_3,Z_2$.

	$S_4>A_4> $
\end{enumerate}
%1238,12
\prob \textit{Page 111,2}
	If $G=G_0>G_1>\cdots>G_n$ is a subnormal series of a finite group $G$, then 
	$|G|=\left(\prod\limits_{i=0}^{n-1}|G_i/G_{i+1}|\right)|G_n|$.
	\label{seriprod}
\soln The conclusion is proved by induction and Lagrange Theorem.
	\begin{itemize}
		\item If $n=1$, $|G|=|G_0/G_1||G_1|$ follows from Lagrange Theorem.
		\item It suffices to prove 
	$|G|=\left(\prod\limits_{t=0}^{n}|G_i/G_{i+1}|\right)|G_{n+1}|$ if 
	$|G|=\left(\prod\limits_{t=0}^{n-1}|G_i/G_{i+1}|\right)|G_n|$. Since $|G_n|=|G_n/G_{n+1}||G_{n+1}|$, by induction we get the conclusion.
	\end{itemize}
\prob\textit{Page 111,3} If $N$ is a simple normal subgroup of a group $G$ and $G/N$ has a composition series, then $G$ has a composition series.
\soln
Assuming 
$G/N=H_0>H_1>\cdots>H_n$ 
is the composition series, and then 
$G=H_0N>H_1N>\cdots>H_nN>N$ is the composition series of $G$. 

For any $g\in G$, there exist a unique $h_0\in H_0$ and a unique $n\in N$ such that $g\in h_0N$ and $g=h_0 n$; then we get a homorphism between $G$ and $H_0N$, $G=H_0N$.

Let $H_{n+1}=\{e\}$, then $H_{n+1}\lhd H_{n}$ trivially. For $i=0,1,\cdots,n$, $\forall h_0, h_1\in H_{i+1}, h\in H_{i}, n_0,n\in N$, 
\begin{align*}
	(hN)^{-1} h_0N hN &= N h^{-1} h_0N Nh
	\\ &= h^{-1} h_0 h N,
	\\ (h_1N)^{-1} h_0N &= N h_1^{-1} h_0N
	\\ &= h_1^{-1} h_0 h N.
\end{align*}
Therefore $H_{i+1}N\lhd H_{i}N$ since $H_{i+1}\lhd H_{i}$. Because $N$ is a simple group (with no proper normal subgroups) and $H_{i}N / H_{i+1}N=H_i/H_{i+1}$ is simple 
$G=H_0N>H_1N>\cdots>H_nN>N$ is the composition series of $G$.

\prob\textit{Page 112,8} If $H$ and $K$ are solvable subgroups of $G$ with $H\lhd G$, then $HK$ is a solvable subgroup of $G$.
\soln A group is solvable iff it has a subnoral series with every factors abelian. Assuming the series of $H,K$ are 
$H=H_0>H_1>\cdots>H_n$,  
$K=K_0>K_1>\cdots>K_m$ respectively. 

$HK/H\cong K/(H\cap K)$ by second isomorphism Theorem. \newpage

\prob\textit{Page 112,12} Prove the Fundamental Theorem of Arithmetic by applying the Jordan-H\"{o}lder Theorem to the group $Z_n$.
\soln
For any positive integer (except the number $1$) $n\in \mathbb{Z}\setminus\{1\}$, the group $Z_n$ is a finite group. Then $Z_n$ must has a composition series. 

Because every subgroup of $Z_n$ is still a cyclic group, there exists a sequence $\{a_i\}$ such that 
$Z_n=Z_{a_0}>Z_{a_1}>\cdots>Z_{a_m}$ is a composition series. According to Problem \ref{seriprod},
$$n= \left(\prod\limits_{i=0}^{m-1}a_i/a_{i+1}\right)a_m.$$
This representation of $n$ consist only primes since the series is composition and $Z_{a_i}/Z_{a_{i+1}}=Z_{a_i/a_{i+1}}$ is simple iff $a_i/a_{i+1}$ is prime. By applying the Jordan-H\"{o}lder Theorem, the representation is unique.

\end{enumerate}
\end{document}