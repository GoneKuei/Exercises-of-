
\documentclass{article}%{ctexart}
\usepackage{CJKutf8}
%\usepackage[UTF8]{CTeX}
%\usepackage{xeCJK} %调用 xeCJK 宏包
%\setCJKmainfont{SimSun} %设置 CJK 主字体为 SimSun (宋体

\usepackage{amsmath,amssymb,amsthm,color,mathrsfs}
\usepackage{enumerate,anysize}%
\usepackage{qrcode,fancyhdr}
\usepackage{commath}
%\pd,\dif,\od,...
\pagestyle{fancy}
\lhead{201821130049}%\rhead{段奎元}
\marginsize{1in}{1in}{1in}{1in}%

%%Settheory notation 
\theoremstyle{remark}
\newtheorem*{claim*}{Claim}
\newtheorem{lemma}{Lemma}
\DeclareMathOperator{\ot}{ordertype}
\DeclareMathOperator{\dom}{dom}
\DeclareMathOperator{\ran}{ran}
\DeclareMathOperator{\Ord}{Ord}
%\DeclareMathOperator{\sup}{sup}
%\DeclareMathOperator{\inf}{inf}

%\def\a#1{\mathbb{#1}}%代数常用的双体
\def\p#1{\mathscr{#1}} %概率论常用的手写体
\def\ae{\textrm{ a.e.}} \def\aeto{\xrightarrow{\ae}}
\def\pp{\mathbb{P}}
%微分几何符号用commath包的命令
%\def\t{T\!\!}%微分几何用切空间
%\def\d{\,\mathrm{d}}
\def\r{\mathbb{R}}
\newcommand{\cc}{\mathfrak{c}}
\def\pow#1{\mathscr P #1}

\def\<{\langle}
\def\>{\rangle}
\def\res{\!\!\upharpoonright\!\!_}%refine the restriction
\def\bcup{\textstyle\bigcup}
\def\bcap{\textstyle\bigcap}
\def\bsum{\textstyle\sum}
\def\bprod{\textstyle\prod}
\def\Oset{\varnothing}
\def\diff{\mathbin{{\Delta}}}

\def\email#1{{\texttt{#1}}}
\def\prob{\par\color{blue}\item}
\def\soln{\color{black}\par\noindent\underline{\sc Solution}:\hspace*{1em}\parindent=2em}
\newcommand{\isep}[1][0pt]{\addtolength{\itemsep}{#1}}

\AtBeginDocument{\begin{CJK}{UTF8}{gbsn}}
\AtEndDocument{\end{CJK}}
%%%重新使用cjkutf8,因仅在姓名使用中文,章节定理等仍用英文%
%\begin{document}%在document后定义含中文的命令
\def\asemail{段奎元\\ %% 姓名
	SID: 201821130049\\ %% 学号
	\email{dkuei@outlook.com}} %% 电邮}aUTHORsIDeMAIL
%尝试不用这些命令
%# -*- coding:utf-8 -*-
%! Mode:: "TeX:UTF-8"
%!TEX encoding = UTF-8 Unicode 
%!TEX TS-program = pdflatex
%%%%%%%%
\begin{document}
\title{Algebra HW7}
\author{\asemail}
\maketitle

%按照题号抄题并解答。
\begin{enumerate}
	\isep[1em]

\prob
\textit{Page 148,1 }%35810共5题
Determine the complete ring of quotients of the ring $Z_n$ for each $n\geq 2$.
\soln
For any $r \in R$, $i,j\in J$, $k,l\in K$, 
\begin{align*}
	(ri)a & =r(ia) =r0=0, \\
	(i-j)a & =ia-ja=0-0=0; \\
	a(kr) & =(ak)r =0r=0, \\
	a(k-l) & =ak-al=0-0=0. 
\end{align*}
Thus $ri \in J, i-j \in J$, $J$ is a left ideal; $kr \in K, k-l \in K$, $K$ is a right ideal.
\qed

\prob
\textit{Page 133,3}
\begin{enumerate}
	\item The set $E$ of positive even integers is a multiplicative subset of $Z$ such that $E-I(Z)$ is the field of rational numbers.
	\item State and prove condition(s) on a multiplicative subset S of Z which insure
that S-IZ is the field of rationals.
\end{enumerate}
\soln
\begin{enumerate}
	\item For any $a,b\in E$, $ab$ is still positive and even. Then $E$ is a multiplicative subset of $Z$.
\end{enumerate}
\qed

\prob
\textit{Page 148,5}
Let $R$ be an integral domain with quotient field $F$. 
If $T$ is an integral domain such that $R\subset T \subset F$, then $F$ is (isomorphic to) the quotient field of $T$.
\soln

\prob
\textit{Page 148,8}
Let $R$ be a commutative ring with identity, $I$ an ideal of $R$ and $\pi: R\to R/I$ the canonical projection.
\begin{enumerate}
	\item If $S$ is a multiplicative subset of $R$, then $\pi S = \pi(S)$ is a multiplicative subset of $R/ I$.
	\item The mapping $\theta : S^{-1} R \to (\pi S)^{-1}(R/ I)$ given by $r/ s f\mapsto \pi(r)/\pi(s)$ is a welldefined function.
	\item $\theta$ is a ring epimorphism with kernel $S^{-1}I$ and hence induces a ring isomorphism $S^{-1}R/S^{-1}I\cong (\pi S)^{-1}(R/ I)$.
\end{enumerate}
\soln

\prob\textit{Page 148,10}
Let $R$ be an integral domain and for each maximal ideal $M$ (which is also prime, of course), consider $R_M$ as a subring of the quotient field of $R$.
Show that $\cap R_M = R$, where the intersection is taken over all maximal ideals $M$ of $R$.
\soln

%$|\frac{x^m-1}{x-1}-m|<\varepsilon$
\end{enumerate}
\end{document}