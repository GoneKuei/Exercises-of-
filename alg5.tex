
\documentclass{article}%{ctexart}
\usepackage{CJKutf8}
%\usepackage[UTF8]{CTeX}
%\usepackage{xeCJK} %调用 xeCJK 宏包
%\setCJKmainfont{SimSun} %设置 CJK 主字体为 SimSun (宋体

\usepackage{amsmath,amssymb,amsthm,color,mathrsfs}
\usepackage{enumerate,anysize}%
\usepackage{qrcode,fancyhdr}
\usepackage{commath}
%\pd,\dif,\od,...
\pagestyle{fancy}
\lhead{201821130049}%\rhead{段奎元}
\marginsize{1in}{1in}{1in}{1in}%

%%Settheory notation 
\theoremstyle{remark}
\newtheorem*{claim*}{Claim}
\newtheorem{lemma}{Lemma}
\DeclareMathOperator{\ot}{ordertype}
\DeclareMathOperator{\dom}{dom}
\DeclareMathOperator{\ran}{ran}
\DeclareMathOperator{\Ord}{Ord}
%\DeclareMathOperator{\sup}{sup}
%\DeclareMathOperator{\inf}{inf}

%\def\a#1{\mathbb{#1}}%代数常用的双体
\def\p#1{\mathscr{#1}} %概率论常用的手写体
\def\ae{\textrm{ a.e.}} \def\aeto{\xrightarrow{\ae}}
\def\pp{\mathbb{P}}
%微分几何符号用commath包的命令
%\def\t{T\!\!}%微分几何用切空间
%\def\d{\,\mathrm{d}}
\def\r{\mathbb{R}}
\newcommand{\cc}{\mathfrak{c}}
\def\pow#1{\mathscr P #1}

\def\<{\langle}
\def\>{\rangle}
\def\res{\!\!\upharpoonright\!\!_}%refine the restriction
\def\bcup{\textstyle\bigcup}
\def\bcap{\textstyle\bigcap}
\def\bsum{\textstyle\sum}
\def\bprod{\textstyle\prod}
\def\Oset{\varnothing}
\def\diff{\mathbin{{\Delta}}}

\def\email#1{{\texttt{#1}}}
\def\prob{\par\color{blue}\item}
\def\soln{\color{black}\par\noindent\underline{\sc Solution}:\hspace*{1em}\parindent=2em}
\newcommand{\isep}[1][0pt]{\addtolength{\itemsep}{#1}}

\AtBeginDocument{\begin{CJK}{UTF8}{gbsn}}
\AtEndDocument{\end{CJK}}
%%%重新使用cjkutf8,因仅在姓名使用中文,章节定理等仍用英文%
%\begin{document}%在document后定义含中文的命令
\def\asemail{段奎元\\ %% 姓名
	SID: 201821130049\\ %% 学号
	\email{dkuei@outlook.com}} %% 电邮}aUTHORsIDeMAIL
%尝试不用这些命令
%# -*- coding:utf-8 -*-
%! Mode:: "TeX:UTF-8"
%!TEX encoding = UTF-8 Unicode 
%!TEX TS-program = pdflatex
%%%%%%%%
\begin{document}
\title{Algebra HW5}
\author{\asemail}
\maketitle

%按照题号抄题并解答。
\begin{enumerate}
	\isep[1em]

\prob
\textit{Page 120,3 }%8,11,13共4题
A ring $R$ such that $a^2=a$ for all $a\in R$ is called a \textbf{Boolean ring}. Prove that every Boolean ring $R$ is commutative and $a+a=0$ for all $a\in R$. 
\soln
For all $a\in R$, $a+a = a+a^2 = a+(-a)^2 = a+(-a) =0$. For any $a,b\in R$,
\begin{align*}
	ab-ba &= ab+ba \\
	&= (a+b)^2-(a^2+b^2) \\
	&= (a+b)-(a+b)= 0.
\end{align*}
The Boolean ring $R$ is commutative.
\qed

\prob
\textit{Page 120,8 }%8,11,13共4题
Let $R$ be the set of all $2\times 2$ matrices over complex field $\mathbb{C}$ of the form
\[\begin{pmatrix}
	z & w \\
	-\bar{w} & \bar{z}
\end{pmatrix}.\]
Then $R$ is a division ring that is isomorphic to the division ring $K$ of real quaternions. \textit{Hint:} The fundamental quaternion units $1,i,j,k$ of $K$ map to the matrices respectively, 
\[\begin{pmatrix}
	1 & 0 \\
	0 & 1
\end{pmatrix},\begin{pmatrix}
	i & 0 \\
	0 & -i
\end{pmatrix},\begin{pmatrix}
	0 & 1 \\
	-1 & 0
\end{pmatrix},\begin{pmatrix}
	0 & i \\
	i & 0
\end{pmatrix}.\]
\soln
For any two elements of $R$, 
\begin{align*}
	\begin{pmatrix}
	z & w \\
-\bar{w} & \bar{z}
	\end{pmatrix}-\begin{pmatrix}
	x & y \\
-\bar{y} & \bar{x}
	\end{pmatrix}&=\begin{pmatrix}
	z-x & w-y \\
-\bar{w}+\bar{y} & \bar{z}-\bar{x}
	\end{pmatrix} \in R, \\
	\begin{pmatrix}
	z & w \\
-\bar{w} & \bar{z}
	\end{pmatrix} \begin{pmatrix}
	x & y \\
-\bar{y} & \bar{x}
	\end{pmatrix}&=\begin{pmatrix}
	xz-w\bar{y} & yz+w\bar{x} \\
-\bar{y}\bar{z}-\bar{w}x & \bar{x}\bar{z}-\bar{w}y
	\end{pmatrix} \in R.
\end{align*}
Therefore $R$ is a ring under addition and multiplication of $2\times 2$ matrices. Because
\begin{align*}
	\begin{pmatrix}
		z & w \\
		-\bar{w} & \bar{z}
	\end{pmatrix} \frac{1}{z\bar{z}+w\bar{w}}
	\begin{pmatrix}
		\bar{z} & -w \\
		\bar{w} & z
	\end{pmatrix} = 
	\begin{pmatrix}
		1 & 0 \\
		0 & 1
	\end{pmatrix}, 
\end{align*}
$R$ is a division ring.

Construct $\phi$ mapping the fundamental quaternion units $1,i,j,k$ of $K$ to the matrices  
\[\begin{pmatrix}
	1 & 0 \\
	0 & 1
\end{pmatrix},\begin{pmatrix}
	i & 0 \\
	0 & -i
\end{pmatrix},\begin{pmatrix}
	0 & 1 \\
	-1 & 0
\end{pmatrix},\begin{pmatrix}
	0 & i \\
	i & 0
\end{pmatrix}\]
respectively. It suffices to verify the multiplication between the matrices is the same as the real quaternions'. $\phi(1)=\begin{pmatrix}
	1 & 0 \\
	0 & 1
\end{pmatrix}$ is the identity of multiplication. 
\begin{align*}
	\phi(i)^2=\begin{pmatrix}
	-1 & 0 \\
	0 & -1
	\end{pmatrix}=\phi(-1), \phi(j)^2=\begin{pmatrix}
	-1 & 0 \\
	0 & -1
	\end{pmatrix}=\phi(-1), \phi(k)^2=\begin{pmatrix}
	-1 & 0 \\
	0 & -1
	\end{pmatrix}=\phi(-1) \\
\end{align*}
\begin{align*}
	\phi(i)\phi(j)=\begin{pmatrix}
	0 & i \\
	i & 0
	\end{pmatrix}=\phi(k),&\quad \phi(j)\phi(i)=\begin{pmatrix}
	0 & -i \\
	-i & 0
	\end{pmatrix}=-\phi(k)=\phi(-k), \\
	\phi(j)\phi(k)=\begin{pmatrix}
	i & 0 \\
	0 & -i
	\end{pmatrix}=\phi(i),&\quad \phi(k)\phi(j)=\begin{pmatrix}
	-i & 0 \\
	0 & i
	\end{pmatrix}=-\phi(i)=\phi(-i), \\
	\phi(k)\phi(i)=\begin{pmatrix}
	0 & 1 \\
	-1 & 0
	\end{pmatrix}=\phi(j),&\quad \phi(i)\phi(k)=\begin{pmatrix}
	0 & -1 \\
	1 & 0
	\end{pmatrix}=-\phi(j)=\phi(-j). \\
\end{align*}
Thus $\phi$ is a homeomorphism. And for any element of $R$, 
\begin{align*}
	\begin{pmatrix}
		a+bi & c+di \\
		-c+di & a-bi
	\end{pmatrix}&= 
	a\phi(1)+b\phi(i)+c\phi(j)+d\phi(k), 
\end{align*}
it is generated by the four matrices. Thus $\phi$ is an isomorphism from $R$ to the division ring $K$.
\qed

\prob
\textit{Page 120,11(The Freshman's Dream). }%8,11,13共4题
Let $R$ be a commutative ring with identity of prime characteristic $p$. If $a,b\in R$, then $(a\pm b)^{p^n}=a^{p^n}\pm b^{p^n}$ for all integers $n\geq 0$. [Note that $b=-b$ if $p=2$.]
\soln
Use the \textit{Binomial Theorem} and $\binom{p^n}{k}$ divisible by $p$ for $1\leq k\leq p^n-1$,  
\begin{align*}
	(a+ b)^{p^n} &= \sum_{k=0}^{p^n} \binom{p^n}{k} a^k b^{p^n-k} \\
	&= \sum_{k=1}^{p^n-1} \binom{p^n}{k} a^k b^{p^n-k} + a^0  b^{p^n}+a^{p^n} b^0 \\
	&= a^{p^n}+ b^{p^n}.
\end{align*}
Substitude $b$ by $-b$, we get $(a- b)^{p^n}=a^{p^n}+ (-b)^{p^n}=a^{p^n}+ (-1)^{p^n}b^{p^n}$. If $p\neq 2$, $(-1)^{p^n}=-1$; if $p=2$, $(-b)^{p^n}=b^{p^n-1}(-b)=-b^{p^n}$. In both cases, $(a- b)^{p^n}=a^{p^n}- b^{p^n}$.
\qed

\prob
\textit{Page 120,13 }%8,11,13共4题
In a ring $R$ the following conditions are equivalent.
\begin{enumerate}
	\item $R$ has no nonzero nilpotent elements.
	\item If $a\in R$ and $a^2=0$, then $a=0$.
\end{enumerate}
\soln

\begin{enumerate}
	\item If $a\in R$ and $a^2=0$, $a$ is a nilpotent element. 
	Since $R$ has no nonzero nilpotent elements, $a=0$.
	\item If $a\in R$ is a nilpotent element and $a^n=0$; 
	If $n=1,2,a=0$. If $n> 2$, $a^{2(n-1)}=a^{n-2}a^n=0$, then $a^{n-1}=0$. 
	Recursively, we get $a^2=0$ and then $a=0$. So any nilpotent element of $R$ is zero.
\end{enumerate}	
\qed

\end{enumerate}
\end{document}