
\documentclass{article}%{ctexart}
\usepackage{CJKutf8}
%\usepackage[UTF8]{CTeX}
%\usepackage{xeCJK} %调用 xeCJK 宏包
%\setCJKmainfont{SimSun} %设置 CJK 主字体为 SimSun (宋体

\usepackage{amsmath,amssymb,amsthm,color,mathrsfs}
\usepackage{enumerate,anysize}%
\usepackage{qrcode,fancyhdr}
\usepackage{commath}
%\pd,\dif,\od,...
\pagestyle{fancy}
\lhead{201821130049}%\rhead{段奎元}
\marginsize{1in}{1in}{1in}{1in}%

%%Settheory notation 
\theoremstyle{remark}
\newtheorem*{claim*}{Claim}
\newtheorem{lemma}{Lemma}
\DeclareMathOperator{\ot}{ordertype}
\DeclareMathOperator{\dom}{dom}
\DeclareMathOperator{\ran}{ran}
\DeclareMathOperator{\Ord}{Ord}
%\DeclareMathOperator{\sup}{sup}
%\DeclareMathOperator{\inf}{inf}

%\def\a#1{\mathbb{#1}}%代数常用的双体
\def\p#1{\mathscr{#1}} %概率论常用的手写体
\def\ae{\textrm{ a.e.}} \def\aeto{\xrightarrow{\ae}}
\def\pp{\mathbb{P}}
%微分几何符号用commath包的命令
%\def\t{T\!\!}%微分几何用切空间
%\def\d{\,\mathrm{d}}
\def\r{\mathbb{R}}
\newcommand{\cc}{\mathfrak{c}}
\def\pow#1{\mathscr P #1}

\def\<{\langle}
\def\>{\rangle}
\def\res{\!\!\upharpoonright\!\!_}%refine the restriction
\def\bcup{\textstyle\bigcup}
\def\bcap{\textstyle\bigcap}
\def\bsum{\textstyle\sum}
\def\bprod{\textstyle\prod}
\def\Oset{\varnothing}
\def\diff{\mathbin{{\Delta}}}

\def\email#1{{\texttt{#1}}}
\def\prob{\par\color{blue}\item}
\def\soln{\color{black}\par\noindent\underline{\sc Solution}:\hspace*{1em}\parindent=2em}
\newcommand{\isep}[1][0pt]{\addtolength{\itemsep}{#1}}

\AtBeginDocument{\begin{CJK}{UTF8}{gbsn}}
\AtEndDocument{\end{CJK}}
%%%重新使用cjkutf8,因仅在姓名使用中文,章节定理等仍用英文%
%\begin{document}%在document后定义含中文的命令
\def\asemail{段奎元\\ %% 姓名
	SID: 201821130049\\ %% 学号
	\email{dkuei@outlook.com}} %% 电邮}aUTHORsIDeMAIL
%尝试不用这些命令
%# -*- coding:utf-8 -*-
%! Mode:: "TeX:UTF-8"
%!TEX encoding = UTF-8 Unicode 
%!TEX TS-program = pdflatex
%%%%%%%%
\begin{document}
\title{Differential Geometry HW3}
\author{\asemail}
\maketitle

%按照题号抄题并解答。
\begin{enumerate}
\isep[1em]

\prob A tensor is \textbf{decomposable} if it is the tensor product of several vectors. Is every tensor decomposable? Prove or give a counter example.
\soln
No. Assume that $V^*$ is a $2$-dimensional dual vector space over $\r$ with a basis $\{e_1,e_2\}$. 
For a tensor $v\otimes w\in V^*\otimes V^*$, there exist $x\in \r^4$ such that $$
v\otimes w=(e_1\otimes e_1, e_1\otimes e_2, e_2\otimes e_1, e_2\otimes e_2)\cdot x .$$
If $v\otimes w$ is decomposable, $\exists a_1 e_1+ a_2 e_2, b_1 e_1+ b_2 e_2\in V^*$ such that 
\begin{align*}
	v\otimes w &= (a_1 e_1+ a_2 e_2)\otimes (b_1 e_1+ b_2 e_2) \\
	&= a_1b_1 e_1\otimes e_1+a_1b_2 e_1\otimes e_2+a_2b_1 e_2\otimes e_1+a_2b_2 e_2\otimes e_2 \\
	&= (e_1\otimes e_1, e_1\otimes e_2, e_2\otimes e_1, e_2\otimes e_2)\cdot (a_1b_1, a_1b_2, a_2b_1, a_2b_2)^T.
\end{align*}
Then $x=(a_1b_1, a_1b_2, a_2b_1, a_2b_2)^T$. Every tensor is decomposable means every $x\in \r^4$ has the later form; this is not true. For example, $x=(0,1,1,1)$, then $a_1b_1=0$ infers $a_1=0$ or $b_1=0$; hence $a_1b_2=0$ or $a_2b_1=0$, but $x_2=x_3=1\neq 0$. Contradiction!
\qed

\prob Calculate the induce metric of an ellipsoid surface, $$
S^n_2=\{x\in \r^{n+1}\mid 4x_0^2+\sum_{i=1}^n x_i^2=1 \}\hookrightarrow \r^{n+1}. $$
\soln
Denote $N=(1/2,0,\cdots,0), S=(-1/2,0,\cdots,0)\in S^n_2$, $V_1=S^n_2\setminus \{N\}, V_2=S^n_2\setminus \{S\}$. $\{V_1,V_2\}$ is an open cover of $S^n_2$. Define the stereographic projection $X:\r^n \to V_1, Y:\r^n \to V_2,$ 
\begin{align*}
	X(\vec{x})& =\frac{1}{1+\vec{x}^2}(\frac{\vec{x}^2-1}{2},2\vec{x}), \\
	Y(\vec{x})& =\frac{1}{1+\vec{y}^2}(\frac{1-\vec{y}^2}{2},2\vec{y})
\end{align*}
where $\vec{x}\in \r^n,(a,2\vec{x})\in \r^{n+1}$ for any $a\in \r$, $\vec{y}=\frac{\vec{x}}{\vec{x}^2}$.

Denote the induce metrics of $(X,\r^n), (Y,\r^n)$ as $g^+,g^-$ respectively, the tagent vector $\frac{\partial}{\partial x_\alpha} \in T\r^{n+1}$ as $e_\alpha=(0,\cdots,1,\cdots,0)$ (the $\alpha$-th component is $1$ and all other components are $0$) for $0\leq\alpha\leq n$. Then in $V_1=X(\r^n)$, 
\begin{align*}
	\d X(\frac{\partial}{\partial x_i}) &= \frac{\partial X}{\partial x_i} \\
	& = \frac{1}{(\vec{x}^2+1)^2}(2x_i, 2\delta_{1i}(\vec{x}^2+1)-4x_1 x_i,\cdots, 2\delta_{ni}(\vec{x}^2+1)-4x_n x_i)
\end{align*}
for $1\leq i\leq n$. Then
\begin{align*}
	g^+_{ij} & =\<\frac{\partial}{\partial x_i},\frac{\partial}{\partial x_j}\>_X 
	=\<\d X(\frac{\partial}{\partial x_i}), \d X(\frac{\partial}{\partial x_j})\>_{\r^{n+1}} 
	= \<\frac{\partial X}{\partial x_i}, \frac{\partial X}{\partial x_i}\> \\
	& = \frac{4x_ix_j}{(\vec{x}^2+1)^4} +\sum_{k=1}^n\frac{(2\delta_{ki}(\vec{x}^2+1)-4x_k x_i)(2\delta_{kj}(\vec{x}^2+1)-4x_k x_j)}{(\vec{x}^2+1)^4} \\
	& = \frac{4\delta_{ij}(\vec{x}^2+1)^2-12x_ix_j}{(\vec{x}^2+1)^4}.
\end{align*}
Hence we get 
\begin{align*}
	g^+ &= g^+_{ij} \d x_i \d x_j \\
	& =\frac{4\delta_{ij}(\vec{x}^2+1)^2-12x_ix_j}{(\vec{x}^2+1)^4} \d x_i \d x_j .
\end{align*} Similarly, 
\begin{align*}
	g^- &= \frac{4\delta_{ij}(\vec{y}^2+1)^2-12 y_iy_j}{(\vec{y}^2+1)^4} \d y_i \d y_j.
\end{align*}
On $V_1 \cap V_2$, 
\begin{align*}
	g^- &= \frac{4\delta_{ij}(\vec{y}^2+1)^2-12 y_iy_j}{(\vec{y}^2+1)^4} \d y_i \d y_j \\
	&= \frac{4\delta_{ij}(1/\vec{x}^2+1)^2-12 x_ix_j/\vec{x}^4}{(1/\vec{x}^2+1)^4 \vec{x}^4} \d x_i \d x_j \\
	&= \frac{4\delta_{ij}(\vec{x}^2+1)^2-12x_ix_j}{(\vec{x}^2+1)^4} \d x_i \d x_j
\end{align*}
\qed

\end{enumerate}
\end{document}