
\documentclass{article}%{ctexart}
\usepackage{CJKutf8}
%\usepackage[UTF8]{CTeX}
%\usepackage{xeCJK} %调用 xeCJK 宏包
%\setCJKmainfont{SimSun} %设置 CJK 主字体为 SimSun (宋体

\usepackage{amsmath,amssymb,amsthm,color,mathrsfs}
\usepackage{enumerate,anysize}%
\usepackage{qrcode,fancyhdr}
\usepackage{commath}
%\pd,\dif,\od,...
\pagestyle{fancy}
\lhead{201821130049}%\rhead{段奎元}
\marginsize{1in}{1in}{1in}{1in}%

%%Settheory notation 
\theoremstyle{remark}
\newtheorem*{claim*}{Claim}
\newtheorem{lemma}{Lemma}
\DeclareMathOperator{\ot}{ordertype}
\DeclareMathOperator{\dom}{dom}
\DeclareMathOperator{\ran}{ran}
\DeclareMathOperator{\Ord}{Ord}
%\DeclareMathOperator{\sup}{sup}
%\DeclareMathOperator{\inf}{inf}

%\def\a#1{\mathbb{#1}}%代数常用的双体
\def\p#1{\mathscr{#1}} %概率论常用的手写体
\def\ae{\textrm{ a.e.}} \def\aeto{\xrightarrow{\ae}}
\def\pp{\mathbb{P}}
%微分几何符号用commath包的命令
%\def\t{T\!\!}%微分几何用切空间
%\def\d{\,\mathrm{d}}
\def\r{\mathbb{R}}
\newcommand{\cc}{\mathfrak{c}}
\def\pow#1{\mathscr P #1}

\def\<{\langle}
\def\>{\rangle}
\def\res{\!\!\upharpoonright\!\!_}%refine the restriction
\def\bcup{\textstyle\bigcup}
\def\bcap{\textstyle\bigcap}
\def\bsum{\textstyle\sum}
\def\bprod{\textstyle\prod}
\def\Oset{\varnothing}
\def\diff{\mathbin{{\Delta}}}

\def\email#1{{\texttt{#1}}}
\def\prob{\par\color{blue}\item}
\def\soln{\color{black}\par\noindent\underline{\sc Solution}:\hspace*{1em}\parindent=2em}
\newcommand{\isep}[1][0pt]{\addtolength{\itemsep}{#1}}

\AtBeginDocument{\begin{CJK}{UTF8}{gbsn}}
\AtEndDocument{\end{CJK}}
%%%重新使用cjkutf8,因仅在姓名使用中文,章节定理等仍用英文%
%\begin{document}%在document后定义含中文的命令
\def\asemail{段奎元\\ %% 姓名
	SID: 201821130049\\ %% 学号
	\email{dkuei@outlook.com}} %% 电邮}aUTHORsIDeMAIL
%尝试不用这些命令
%# -*- coding:utf-8 -*-
%! Mode:: "TeX:UTF-8"
%!TEX encoding = UTF-8 Unicode 
%!TEX TS-program = pdflatex
%%%%%%%%
\begin{document}
\title{Probability Theory HW4}
\author{\asemail}
\maketitle

%按照题号抄题并解答。
\begin{enumerate}
	\isep[1em]
\prob
Does any distribution function never decreases? Prove or give a counter example.
\soln
No, the distribution functions never decrease.
\begin{proof}
	For any $a\leq b$, $F(b)$\vspace{6cm}
\end{proof}

\prob
The measure which has finite value on finite interval of $(\r^n, \p B^n)$ is called L-S measure. Prove that every L-S measure is a Lebesgue-Stieljes measure generated by some distribution function.
\soln
	Set $\mu$ an L-S measure and construct the function $F$ on $\r^n$.
	Let $F(0)=0$, $F(x)=\mu([0,x])$ for $x\geq 0$. 
	For other $x=(x_1,x_2,\cdots,x_n)\in \r^n$, let $\bar{x}=(|x_1|,|x_2|,\cdots,|x_n|)\geq 0$. Let $F(x)=F(\bar{x})$ and we get a function on $\r^n$.

	$F$ is continue since $\mu$ is continue for the endpoints. For $a\leq b$, $\Delta_{b,a}F=$
\newpage
	
\prob
If $F(x)=\mathbb{P}(\xi < x)$ is continue, then $\eta = F(\xi)$ has a uniform distribution on $(0,1)$.
\soln
\medskip \vspace{5cm}

\prob
Are the chararcteristic functions and simple functions measurable? Prove or give counter examples.
\soln
For any chararcteristic function $1_A$ where $A \subset \Omega$,
\begin{align*}
	\sigma(1_A) & =\sigma(\{E(x)=\{1_A(\xi)<x \mid \xi \in \Omega\} \mid x\in \Omega\})
	\\ &=\sigma (\{\Oset, A^c, \Omega\})=\sigma(\{A\}).
\end{align*}
Then $\sigma(1_A) \subset \p A$ if and only if $A\in \p A$.
The chararcteristic function $1_A$ is measurable if and only if $A$ is measurable.

For any simple function $f=\sum_{k=1}^n a_k 1_{A_k}$, assume $a_1 \leq a_2 \leq \cdots \leq a_n$.
\begin{align*}
	\sigma(\sum_{k=1}^n a_k 1_{A_k}) & =\sigma(\{E(x)=\{\sum_{k=1}^n a_k 1_{A_k}(\xi)<x \mid \xi \in \Omega\} \mid x\in \Omega\})
	\\ &\subset \sigma (\{\Oset, A_1, A_1\cup A_2, \cdots , \sum_{k=1}^n A_k\})
	\\ &\subset \p A.
\end{align*}
Since every $A_k, k=1,2,\cdots,n$ is in $\p A$, $\sigma(f)\subset \p A$ is measurable.

\end{enumerate}
\end{document}